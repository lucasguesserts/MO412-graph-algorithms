\documentclass{article}
\usepackage[a4paper]{geometry}
\usepackage[english]{babel}
\usepackage[utf8]{inputenc}
\usepackage{url}
\usepackage{hyperref}
\usepackage{graphicx}
\usepackage{amsmath}
\usepackage{amsfonts}
\usepackage{amssymb}
\usepackage{amsthm}
\usepackage{float}
\usepackage{lscape}

\title{Activity 10 - Robustness}
\author{Lucas Guesser Targino da Silva - RA: 203534}

\begin{document}

\maketitle

The results are in the Table \ref{table:results}. There, $f_c$ is the relative\footnote{Relative to the number of nodes of the network.} number of nodes that have to be removed from the network for it to lose its giant component. It is calculated using the Equation \ref{eq:fc}. ``-'' values are fields which are not appliable for the particular network.

For directed networks, the values $f_{c, in}, f_{c, out}, f_{c, directed}$ are provided. They are similar to $f_c$: $f_{c, in}, f_{c, out}$ use $\langle k_{in}^2 \rangle$ and $\langle k_{out}^2 \rangle$ instead of $\langle k^2 \rangle$, respectively; $f_{c, di}$ is calculated according to the Equation \ref{eq:fcdi}.

In the Equation \ref{eq:fcdi}, one chose to use $\max$ because one considers the network to be disconnected when both the ``in'' and ``out'' network are disconnected. Such choice doesn't matter much because, for all networks, both $f_{c, in}, f_{c, out}$ are very close.

\begin{equation}
    \label{eq:fc}
    f_c = 1 - \dfrac{1}{\frac{\langle k^2 \rangle}{\langle k \rangle} - 1}
\end{equation}

\begin{equation}
    \label{eq:fcdi}
    f_{c, di} = \max \{f_{c, in}, f_{c, out}\}
\end{equation}

\begin{table}[H]
\caption{networks and their properties.}
\label{table:results}
\begin{tabular}{lrrrrrrrr}
network & $\langle k \rangle$ & $\langle k_{in}^2 \rangle$ & $\langle k_{out}^2 \rangle$ & $\langle k^2 \rangle$ & $f_{c}$ & $f_{c, in}$ & $f_{c, out}$ & $f_{c, di}$ \\
internet & 6.340 & - & - & 240.100 & 0.973 & - & - & - \\
www & 4.600 & 1546.000 & 482.400 & - & - & 0.997 & 0.990 & 0.997 \\
power grid & 2.670 & - & - & 10.300 & 0.650 & - & - & - \\
mobile phone calls & 2.510 & 12.000 & 11.700 & - & - & 0.736 & 0.727 & 0.736 \\
email & 1.810 & 94.700 & 1163.900 & - & - & 0.981 & 0.998 & 0.998 \\
science collaboration & 8.080 & - & - & 178.200 & 0.953 & - & - & - \\
actor network & 83.710 & - & - & 47353.700 & 0.998 & - & - & - \\
citation network & 10.430 & 971.500 & 198.800 & - & - & 0.989 & 0.945 & 0.989 \\
e. coli metabolism & 5.580 & 535.700 & 396.700 & - & - & 0.989 & 0.986 & 0.989 \\
portein interactions & 2.900 & - & - & 32.300 & 0.901 & - & - & - \\
\end{tabular}
\end{table}


\end{document}
